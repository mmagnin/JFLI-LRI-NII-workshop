% Exemples

% Exemple des définitions + points fixes
\def \exdef {
\TSort{(0,0)}{z}{3}{l}
\TSort{(3,3)}{b}{2}{t}
\TSort{(6,0)}{a}{2}{r}

\THit{b_0}{}{z_1}{.east}{z_2}
\THit{b_1}{}{z_0}{.east}{z_2}
\THit{a_0}{}{b_1}{.south}{b_0}
\THit{a_1}{out=60,in=0,selfhit}{a_1}{.east}{a_0}

\path[bounce,bend right]
\TBounce{z_1}{}{z_2}{.south}
\TBounce{z_0}{bend right=50}{z_2}{.south east}
;
\path[bounce,bend left]
\TBounce{a_1}{}{a_0}{.north}
\TBounce{b_1}{}{b_0}{.south}
;
}

% Idem réorganisé pour Points Fixes
\def \exdefb {
\path[use as bounding box] (0,-1) rectangle (4,4);

\TSort{(0,0)}{z}{3}{l}
\TSort{(2,4)}{b}{2}{t}
\TSort{(4,1)}{a}{2}{r}
}

% Frappes
\def \exdefbfrappes {
\THit{b_0}{}{z_1}{.east}{z_2}
\THit{b_1}{}{z_0}{.east}{z_2}
\THit{a_0}{}{b_1}{.south}{b_0}
\THit{a_1}{out=60,in=0,selfhit}{a_1}{.east}{a_0}

\path[bounce,bend right]
\TBounce{z_1}{}{z_2}{.south}
\TBounce{z_0}{bend right=50}{z_2}{.south east}
;
\path[bounce,bend left]
\TBounce{a_1}{}{a_0}{.north}
\TBounce{b_1}{}{b_0}{.south}
;
}

% Non-frappes
\def \exdefbsf {
\path[use as bounding box] (0,-1) rectangle (4,4);
\node[process,draw=red,thick] (a_1) at (a_1.center) {};

\path<2,4-> (z_0) edge (b_0) edge (a_0) (b_0) edge (a_0);
\path<2-3> (z_2) edge (b_0) edge (a_0);
\path<3>[very thick] (z_0) edge (b_0) edge (a_0) (b_0) edge (a_0);
\path<4->[very thick] (z_2) edge (b_0) edge (a_0) (b_0) edge (a_0);
\TState{3}{z_0,b_0,a_0}
\TState{4-}{z_2,b_0,a_0}

\path (z_2) edge (b_1);
\path (z_1) edge (b_1);
\path (z_1) edge (a_0);
}



% Figure de présentation de l'analyse d'atteignabilité
\def \figsa {
\begin{tikzpicture}
\path[use as bounding box] (-5,-2.6) rectangle (5,2.8);
\definecolor{r2}{RGB}{238,10,38}

\path<2->[shading=1, inner color=r2, outer color=white] (3.5,-2.8) -- (4.4,3.2) -- (0,3) -- (-4.5,1.4) -- (-2.5,-2.5) -- (0,-3.6) -- (2.8,-2.8);
%\path<2->[shading, inner color=r2, outer color=white, border color=white] (2.8,-2.8) -- (4.5,4.5) -- (0,3.9) -- (-4.5,1.8) -- (-5,-3) -- (0,-3.2) -- (2.8,-2.8);
\draw<2->[thick,fill=white] (2.5,-2.1) -- (3,2.5) -- (-2.7,1.3) -- (-2,-2) -- (2.5,-2.1);
\draw<6->[thick,fill=lightyellow] (2.5,-2.1) -- (3,2.5) -- (-2.7,1.3) -- (-2,-2) -- (2.5,-2.1);

\node<2->[text width=3.5cm, color=red] (s1) at (-5,2) {Over-Approximation};
\path<2->[->,very thick,color=red] (s1.south) edge (-3.5,1.2);
%\node<2->[text width=3cm,color=black] (i1) at (3.7,.2) {$\Rightarrow$};
\node<2->[text width=3cm,color=black] (q) at (4.5,.2) {$\neg Q$};

\draw<4->[thick, fill=green] (.5,-.8) -- (1,0) -- (.3,1) -- (-1,.5) -- (-.5,-.5) -- (.5,-.8);
\node<4->[text width=3.5cm,color=darkgreen] (s2) at (5.2,-1.5) {Under-Approximation};
\node<4->[text width=3cm,color=black] (p) at (1.8,.2) {$P$};
%\node<4->[text width=3cm,color=black] (i1) at (2.25,.2) {$\Rightarrow$};

% reaching set
\node[text width=3cm,color=darkcyan] (s) at (1.8,1.7) {Exact solution};
\node<1->[text width=3cm,color=darkcyan] (s0) at (0,0) {};
\draw[color=darkcyan, thick] (0,0) ellipse (2 and 1.5);
%\path<1>[draw=white] (2.8,-2.8) -- (4.5,4.5) -- (0,3.9) -- (-4.5,1.8) -- (-5,-3) -- (-2.5,-3.5) -- (0,-3.2) -- (2.8,-2.8);
\node[text width=3cm,color=black] (r) at (2.8,.2) {$R$};

\path<4->[->,very thick,color=darkgreen] (s2) edge (.6,-.4);

\tikzstyle{point}=[circle,draw=blue,fill=blue,minimum size=5pt,inner sep=0pt]

%\only<5->{
\only<3->{
\node[point] at (-2.4,-2) {};
\node[point] at (-2,2) {};
}
\only<5->{
\node[point] at (0,0) {};
}
\only<7->{
\node[point] at (-.5,-1.1) {};
\node[point] at (2.5,1) {};
}
%}

\end{tikzpicture}
}



% Exemple atteignabilité
\def \exatt {
\path[use as bounding box] (-1,-3) rectangle (7,2);
\TSort{(0,0)}{a}{2}{l}
\TSort{(3,0)}{b}{3}{l}
\TSort{(6,0)}{d}{3}{r}
\TSort{(2,-2)}{c}{2}{b}

\THit{a_0}{}{c_0}{.north}{c_1}
\THit{a_1}{}{b_1}{.west}{b_0}
\THit{c_1}{bend left=20pt}{b_0}{.west}{b_1}
\THit{b_1.south west}{->}{a_0}{.east}{a_1}
\THit{b_0}{}{d_0}{.west}{d_1}
\THit{b_1}{}{d_1}{.west}{d_2}
\THit{d_1}{}{b_0}{.north east}{b_2}
\THit{c_1}{bend right=80pt,distance=80pt}{d_1}{.east}{d_0}
\THit{b_2}{distance=120pt,out=30,in=40}{d_0}{.east}{d_2}

\path[bounce,bend left]
\TBounce{d_0}{}{d_1}{.south}
\TBounce{d_1}{}{d_2}{.south}
\TBounce{c_0}{}{c_1}{.west}
\TBounce{b_0}{}{b_1}{.south}
\TBounce{d_1}{}{d_0}{.north}
;
\path[bounce,bend right]
\TBounce{a_0}{}{a_1}{.south}
\TBounce{b_0}{}{b_2}{.south}
\TBounce{b_1}{}{b_0}{.north}
\TBounce{d_0}{bend right=50pt,distance=40pt}{d_2}{.south}
;
}



% Structure abstraite / Sous-approximation / Ok
\def \sauyes {%
\begin{tikzpicture}[aS,node distance=1.1cm,shorthandon]
\path[use as bounding box] (-0.5,-2.1) rectangle (10.25,2.2);

\node[Aobj] (d02) {$\PHobjectif{d_0}{d_2}$};
\node[Aproc,above of=d02] (d2) {$d_2$};

\node[Asol,right of=d02] (d02s2) {};
\node[Aproc,above right of=d02s2] (b0) {$b_0$};
\node[Aobj,right of=b0] (b10) {$\PHobjectif{b_1}{b_0}$};
\node[Asol,right of=b10] (b10s) {};
\node[Aproc,right of=b10s] (a1) {$a_1$};
\node[Aobj,right of=a1] (a11) {$\PHobjectif{a_1}{a_1}$};
\node[Asol,right of=a11] (a11s) {};

\node[Aobj,above of=b10,yshift=-0.5cm] (b00)
{$\PHobjectif{b_0}{b_0}$};
\node[Asol,right of=b00] (b00s) {};

\node[Aproc, below of=b0] (b1) {$b_1$};
\node[Aobj,right of=b1] (b11) {$\PHobjectif{b_1}{b_1}$};
\node[Asol,right of=b11] (b11s) {};
\node[Aobj,below of=b11] (b01) {$\PHobjectif{b_0}{b_1}$};
\node[Asol,right of=b01] (b01s) {};
\node[Aproc,right of=b01s] (c1) {$c_1$};
\node[Aobj,right of=c1] (c11) {$\PHobjectif{c_1}{c_1}$};
\node[Asol,right of=c11] (c11s) {};

\path
(d02) edge (d02s2) (d02s2) edge (b1) edge (b0)
(a11) edge (a11s)
(b10) edge (b10s) (b10s) edge (a1)
(b11) edge (b11s)
(b0) edge (b10) (b1) edge (b11)
(a1) edge (a11)
(d2) edge (d02)
;
\path
(b0) edge (b00.west) (b00) edge (b00s)
(b1) edge (b01)
(b01) edge (b01s) (b01s) edge (c1)
(c1) edge (c11) (c11) edge (c11s)
;
%\node<\tu>[right of=a11s] {\textbf{\Large\color{darkgreen}Yes}};
\end{tikzpicture}%
}

% Structure abstraite / Sous-approximation / Inconclusif
\def \sauinconc {%
\begin{tikzpicture}[aS,node distance=1.1cm,shorthandon]
\path[use as bounding box] (-0.5,-2.1) rectangle (10.25,2.2);

\node[Aobj] (d02) {$\PHobjectif{d_0}{d_2}$};
\node[Aproc,above of=d02] (d2) {$d_2$};

\node[Asol,right of=d02] (d02s2) {};
\node[Aproc,above right of=d02s2] (b0) {$b_0$};
\node[Aobj,right of=b0] (b10) {$\PHobjectif{b_1}{b_0}$};
\node[Asol,right of=b10] (b10s) {};
\node[Aproc,right of=b10s] (a1) {$a_1$};
\node[Aobj,right of=a1] (a01) {$\PHobjectif{a_0}{a_1}$};
\node[Asol,right of=a01] (a01s) {};

\node[Aproc, below of=b0] (b1) {$b_1$};
\node[Aobj,right of=b1] (b11) {$\PHobjectif{b_1}{b_1}$};
\node[Asol,right of=b11] (b11s) {};
\node[Aobj,below of=b11] (b01) {$\PHobjectif{b_0}{b_1}$};
\node[Asol,right of=b01] (b01s) {};
\node[Aproc,right of=b01s] (c1) {$c_1$};
\node[Aobj,right of=c1] (c01) {$\PHobjectif{c_0}{c_1}$};
\node[Asol,right of=c01] (c01s) {};
\node[Aproc,right of=c01s] (a0) {$a_0$};
\node[Aobj,right of=a0] (a00) {$\PHobjectif{a_0}{a_0}$};
\node[Asol,right of=a00] (a00s) {};

\node[Aobj,above of=b10] (b00) {$\obj{b_0}{b_0}$};
\node[Asol,right of=b00] (b00s) {};
\node[Aobj,above of=a01] (a11) {$\obj{a_1}{a_1}$};
\node[Asol,right of=a11] (a11s) {};
\node[Aobj,above of=c01] (c11) {$\obj{c_1}{c_1}$};
\node[Asol,right of=c11] (c11s) {};
\node[Aobj,above of=a00] (a10) {$\PHobjectif{a_1}{a_0}$};
\node at (a10.east) {\Large\color{red}\textbf{$\bot$}};

\path
  (b10) edge[loop,min distance=5mm] (b10)
 ;
\path
(d02) edge (d02s2) (d02s2) edge (b1) edge (b0)
(a01) edge (a01s) (a01s.south) edge (b1.north east)
(b10) edge (b10s) (b10s) edge (a1)
(b11) edge (b11s)
(a1) edge (a01)
(b0) edge (b10) (b1) edge (b11)
(d2) edge (d02)
;
\path
(b00) edge (b00s)
(b0) edge (b00)
 (b1) edge (b01)
 (b01) edge (b01s) (b01s) edge (c1)
 (c1) edge (c01)
 (c01) edge (c01s) (c01s) edge (a0)
 (a0) edge (a00) (a00) edge (a00s)
;
\path
 (c1) edge (c11) (c11) edge (c11s)
(a0) edge (a10)
(a1) edge (a11)
(a11) edge (a11s)
;

%\node[right of=a01s] {\textbf{\Large\color{darkyellow}Inconc}};

\end{tikzpicture}%
}

% Structure abstraite / Sur-approximation / Non
\def \saono {%
\begin{tikzpicture}[aS,node distance=1.1cm,shorthandon]
\path[use as bounding box] (-0.5,-2.1) rectangle (10.25,1.15);

\node[Aobj] (d12) {$\PHobjectif{d_1}{d_2}$};
\node[Asol,above right of=d12] (d12s1) {};
\node[Aproc, right of=d12s1] (b2) {$b_2$};
\node[Aobj,right of=b2] (b02) {$\PHobjectif{b_0}{b_2}$};
\node[Asol,right of=b02] (b02s) {};
\node[Aproc,right of=b02s] (d1) {$d_1$};
\node[Aobj,right of=d1] (d11) {$\PHobjectif{d_1}{d_1}$};
\node[Asol,right of=d11] (d11s) {};

\node[Asol,below right of=d12] (d12s2) {};
\node[Aproc, right of=d12s2] (b1) {$b_1$};
\node[Aobj,right of=b1] (b01) {$\PHobjectif{b_0}{b_1}$};
\node[Asol,right of=b01] (b01s) {};
\node[Aproc,right of=b01s] (c1) {$c_1$};
\node[Aobj,right of=c1] (c01) {$\PHobjectif{c_0}{c_1}$};
\node[Asol,right of=c01] (c01s) {};
\node[Aproc,right of=c01s] (a0) {$a_0$};
\node[Aobj,right of=a0] (a10) {$\PHobjectif{a_1}{a_0}$};
\node at (a10.east) {\Large\color{red}\textbf{$\bot$}};

\path
(d12) edge (d12s1) edge (d12s2) (d12s1) edge (b2) edge (c1) (d12s2) edge (b1)
(b01) edge (b01s) (b01s) edge (c1)
(b02) edge (b02s) (b02s) edge (d1)
(c01) edge (c01s) (c01s) edge (a0)
(d11) edge (d11s)
(a0) edge (a10)
(b1) edge (b01)
(b2) edge (b02)
(c1) edge (c01)
(d1) edge (d11)
;
%\only<\value{anim1}>{ \node[above right of=c01s] {\textbf{\Large\color{red}No}};}
\end{tikzpicture}%
}

% Structure abstraite / Sur-approximation / Inconclusif
\def \saoinconc {%
\begin{tikzpicture}[aS,node distance=1.1cm,shorthandon]
\path[use as bounding box] (-0.5,-2.1) rectangle (10.25,1.15);

\node[Aobj] (d02) {$\PHobjectif{d_0}{d_2}$};
\node[Asol,above right of=d02] (d02s1) {};

\node[Aproc, right of=d02s1] (b2) {$b_2$};
\node[Aobj,right of=b2] (b12) {$\PHobjectif{b_1}{b_2}$};
\node[Asol,right of=b12] (b12s) {};
\node[Aproc,right of=b12s] (d1) {$d_1$};
\node[Aobj,right of=d1] (d01) {$\PHobjectif{d_0}{d_1}$};
\node[Asol,right of=d01] (d01s) {};

\node[Asol,below right of=d02] (d02s2) {};
%<-3>
\node<-\tof>[Aproc, right of=d02s2] (b0) {$b_0$};
\node<\tokp>[orange, thick, Aproc, right of=d02s2] (b0) {$b_0$};
\node[Aobj,right of=b0] (b10) {$\PHobjectif{b_1}{b_0}$};
\node[Asol,right of=b10] (b10s) {};
%<-3>
\node<-\tof>[Aproc,right of=b10s] (a1) {$a_1$};
\node<\tokp>[orange, thick, Aproc,right of=b10s] (a1) {$a_1$};
\node[Aobj,right of=a1] (a11) {$\PHobjectif{a_1}{a_1}$};
\node[Asol,right of=a11] (a11s) {};

\node[Aproc, below of=b0] (b1) {$b_1$};
\node[Aobj,right of=b1] (b11) {$\PHobjectif{b_1}{b_1}$};
\node[Asol,right of=b11] (b11s) {};

\node<\tokp>[orange, font=\bfseries,below of=a11s] (kp) {Key processes};
\path<\tokp>[orange, thick]
        (kp) edge (a1)
        (kp) edge (b0)
;
\path
(d02) edge (d02s1) edge (d02s2) (d02s1) edge (b2) (d02s2) edge (b1) edge (b0)
(a11) edge (a11s)
(b10) edge (b10s) (b10s) edge (a1)
(b11) edge (b11s)
(b12) edge (b12s) (b12s) edge (d1) edge (a1)
(d01) edge (d01s) (d01s.south) edge (b0)
(a1) edge (a11)
(b0) edge (b10) (b1) edge (b11) (b2) edge (b12)
(d1) edge (d01)
;
%\node[below right of=d01s] {\textbf{\Large\color{yellow}Inconc}};
\end{tikzpicture}%
}

\frame{\frametitle{Contribution}

\begin{alertblock}{Scientific challenge}
How can we cope with the analysis of \textbf{large-scale systems}, involving up to 1.000 interacting components?
\end{alertblock}

\begin{block}{Objectives of this part}
\begin{itemize}
\item Introduce a Process Algebra inspired framework based on a compact representation of the interactions
\item Develop efficient \textbf{static analysis approaches} to answer most common problems 
\item Apply the methodology to large-scale biological regulatory networks 
\end{itemize}
\end{block}

\begin{block}{Joint work with}
\begin{itemize}
\item L. Paulev� (ETH Zurich), M. Folschette, O. Roux (IRCCyN) 
\item K. Inoue (NII) 
\end{itemize}
\end{block}

}

% Définition du Process Hitting + sortes coopératives

\frame{\frametitle{Intuitive principle of the Process Hitting framework}

\begin{center}
\begin{tabular}{l l p{70mm}}
Process & = & component \texthl{$a$} at level \texthl{$i$}\\
Interaction & = & \texthl{$a$} at level \texthl{$i$} makes \texthl{$b$} at level \texthl{$j$} increase or decrease to level \texthl{$k$}\\
denoted & & \texthl{$\hit{a_i}{b_j}{b_k}$} (hit and bounce)\\
\hline
\end{tabular}

\vfill
{
\begin{definition}[Interaction and Retroaction]\label{def:action}
\emph{Interaction} ($\hit{a_i}{b_j}{b_k}$), where $a_i$ is the level of a process $a$ and $b_j\neq b_k$,

\emph{Retroaction} ($\hit{a_i}{a_i}{a_k}$): when $a_i=b_j$.
\end{definition}
}
\end{center}

}

\begin{frame}[t]
  \frametitle{The Process Hitting modeling}
%  \framesubtitle{\tcite{PMR12-MSCS}}

% 1 : Sortes
\only<1>{
\tikzstyle{process}=[circle,minimum size=15pt,font=\footnotesize,inner sep=1pt]
\tikzstyle{tick label}=[color=white, font=\footnotesize]
\tikzstyle{tick}=[transparent]
\tikzstyle{hit}=[transparent]
\tikzstyle{selfhit}=[transparent, min distance=30pt,curve to]
\tikzstyle{bounce}=[transparent]
\tikzstyle{hlhit}=[transparent]
\begin{center}\scalebox{\scaleex}{
\begin{tikzpicture}
\exphdef
\end{tikzpicture}
}\end{center}
}

% 2 : Processus
\only<2>{
\tikzstyle{process}=[circle,draw,minimum size=15pt,font=\footnotesize,inner sep=1pt]
\tikzstyle{tick label}=[font=\footnotesize]
\tikzstyle{tick}=[densely dotted]
\tikzstyle{hit}=[transparent]
\tikzstyle{selfhit}=[transparent, min distance=30pt,curve to]
\tikzstyle{bounce}=[transparent]
\tikzstyle{hlhit}=[transparent]
\begin{center}\scalebox{\scaleex}{
\begin{tikzpicture}
\exphdef
\end{tikzpicture}
}\end{center}
}

% 3 : États
\only<3>{
\tikzstyle{hit}=[transparent]
\tikzstyle{selfhit}=[transparent, min distance=30pt,curve to]
\tikzstyle{bounce}=[transparent]
\tikzstyle{hlhit}=[transparent]
\begin{center}\scalebox{\scaleex}{
\begin{tikzpicture}
\exphdef

\TState{3}{a_0,b_1,z_0}
\end{tikzpicture}
}\end{center}
}

% 4 : Actions
\only<4->{
\tikzstyle{tick}=[densely dotted]
\tikzstyle{hit}=[->,>=angle 45]
\tikzstyle{selfhit}=[min distance=30pt,curve to]
\tikzstyle{bounce}=[densely dotted,>=stealth',->]
\tikzstyle{hlhit}=[very thick]
\begin{center}\scalebox{\scaleex}{
\begin{tikzpicture}
\exphdef
\TState{4}{a_0,b_1,z_0}
\TState{5}{a_0,b_1,z_1}
\TState{6}{a_1,b_1,z_1}
\TState{7}{a_1,b_1,z_2}
\end{tikzpicture}
}\end{center}
}

\medskip
\begin{liste}
  \item \tval{Sorts}: components \qex{$a$, $b$, $z$}
\pause[2]
  \item \tval{Processes}: local states / levels of expression \qex{$z_0$, $z_1$, $z_2$}
\pause[3]
  \item \tval{States}: sets of active processes%
  \only<3-4>{\qex{$\PHetat{a_0, b_1, z_0}$}}%
  \only<5>{\qex{$\PHetat{a_0, b_1, z_1}$}}%
  \only<6>{\qex{$\PHetat{a_1, b_1, z_1}$}}%
  \only<7>{\qex{$\PHetat{a_1, b_1, z_2}$}}%
\pause[4]
  \item \tval{Actions}: dynamics \qex{\only<4>{\underline}{$\PHfrappe{b_1}{z_0}{z_1}$}, \only<4-5>{\underline}{$\PHfrappe{a_0}{a_0}{a_1}$}, \only<6>{\underline}{$\PHfrappe{a_1}{z_1}{z_2}$}}
\end{liste}
\end{frame}



\begin{frame}
  \frametitle{Adding cooperations}
  \framesubtitle{\cite{PMR12-MSCS}}

\begin{center}\scalebox{\scaleex}{
\begin{tikzpicture}
\exphcoop
\end{tikzpicture}
}\end{center}

\medskip
\only<-14>{
\begin{liste}
  \item How to introduce some \tval{cooperation} between sorts? \qex{$\PHfrappe{a_1 \wedge b_0}{z_1}{z_2}$}
\pause[4]
  \item Solution: a \tval{cooperative sort} \qex{$ab$} \only<12->{\quad to express \qex{$a_1 \wedge b_0$}}
\pause[8]
  \item Constraint: each configuration is represented by one process \qex{$\PHetat{a_1,b_0} \pause[11]\Rightarrow ab_{10}$}
\pause[14]
  \item Advantage: regular sort; drawbacks: complexity, temporal shift
\end{liste}}
\end{frame}



%\subsubsection{Fixed Points}

\begin{frame}[c]
  \frametitle{Static Analysis: Fixed Points}
  \framesubtitle{\cite{PMR10-TCSB}}

\tval{Fixed point} = state where no action can be fired
\begin{fleches}
  \item avoid couples of processes bounded by an action
\only<1>{\\\smallskip~}
\only<2->{\item Hitless Graph \only<3->{\f \tval{n-cliques} = fixed points}}
\end{fleches}

\bigskip
\begin{columns}
\begin{column}{0.5\textwidth}

\begin{center}
\scalebox{0.7}{
\begin{tikzpicture}
\exdefb
\exdefbfrappes
\end{tikzpicture}
}
\end{center}

\end{column}
\begin{column}{0.5\textwidth}

\only<2->{
\begin{center}
\scalebox{0.7}{
\tikzstyle{current process}=[process,fill=red]
\begin{tikzpicture}[hitless graph]
\exdefb
\exdefbsf
\end{tikzpicture}
\tikzstyle{current process}=[process,fill=blue]
}
\end{center}
}

\end{column}
\end{columns}

\bigskip
\pause[6]
Exponential complexity w.r.t.~the number of sorts

\end{frame}

%\begin{frame}
%  \frametitle{Static analysis: successive reachability}
%  \framesubtitle{\cite{PMR12-MSCS}}
%  
%  \begin{alertblock}{Problem}
%  Given an initial state of a Process Hitting, is it possible to reach successively $a_i$, then $b_j$, then $a_k$, then $c_l$, $\ldots$? \\
%  $\Rightarrow$ Combinatorial explosion of the dynamics to explore 
%  \end{alertblock}
%  
%  \begin{block}{Key idea}
%  Instead of checking the successive reachability $\mathcal{R}$, which is complex, we will check: 
%  \begin{itemize}
%  \item an \textbf{under-approximation} $\mathcal{P}$: if $\mathcal{P}$ is not satisfied, then $\mathcal{R}$ neither
%  \item an \textbf{over-approximation} $\mathcal{Q}$: if $\mathcal{Q}$ is satisfied, then $\mathcal{R}$ too. 
%  \end{itemize}
%  \end{block}
%  
%\end{frame}  
%%
%\begin{frame}
%  \frametitle{Static analysis: successive reachability}
%  \framesubtitle{\cite{PMR12-MSCS}}
%
%Successive reachability of processes:
%
%\begin{columns}
%\begin{column}{0.55\textwidth}
%
%\begin{center}
%\scalebox{0.75}{
%\begin{tikzpicture}
%%\path[use as bounding box] (-1,-3) rectangle (7,2);
%\exatt
%
%\TState{2-4}{a_0,b_0,b_2,c_0,d_0}
%
%\TState{5}{a_0,b_0,c_0,d_0}
%\TState{6}{a_0,b_0,c_1,d_0}
%\TState{7}{a_0,b_1,c_1,d_0}
%\TState{8}{a_0,b_1,c_1,d_1}
%\TState{9}{a_0,b_1,c_1,d_2}
%
%\node<3>[process,very thick] (d_1) at (d_1.center) {1?};
%\node<3>[process,very thick] (b_1) at (b_1.center) {2?};
%\node<3>[process,very thick] (d_2) at (d_2.center) {3?};
%
%\node<4-8>[process,very thick] (d_2) at (d_2.center) {1?};
%\node<9>[process,very thick] (d_2) at (d_2.center) {};
%
%\only<5>{\THit{a_0}{hlhit}{c_0}{.north}{c_1}}
%\path<5>[bounce,bend left,hlhit] \TBounce{c_0}{}{c_1}{.west};
%\only<6>{\THit{c_1}{bend left=20pt,hlhit}{b_0}{.west}{b_1}}
%\path<6>[bounce,bend left,hlhit] \TBounce{b_0}{}{b_1}{.south};
%\only<7>{\THit{b_0}{hlhit}{d_0}{.west}{d_1}}
%\path<7>[bounce,bend left,hlhit] \TBounce{d_0}{}{d_1}{.south};
%\only<8>{\THit{b_1}{hlhit}{d_1}{.west}{d_2}}
%\path<8>[bounce,bend left,hlhit] \TBounce{d_1}{}{d_2}{.south};
%\end{tikzpicture}
%}
%\end{center}
%
%\end{column}
%\begin{column}{0.45\textwidth}
%
%\pause
%~\\~\\~\\~\\
%\begin{itemize}
%  \item Initial context
%    \\ \rex{\PHetat{a_1, \{b_0, b_1\}, c_0, d_0}} \pause
%  \item Objectives
%    \\ \rex{$[\ \Rsh d_1 \PHconcat\ \Rsh b_1 \PHconcat\ \Rsh d_2\ ]$} \pause
%    \\\smallskip \rex{$[\ \Rsh d_2\ ]$} \pause
%\end{itemize}
%
%\end{column}
%\end{columns}
%
%\medskip
%\begin{center}
%\f Concretization of the objective = scenario
%
%\ex{
%\only<5>{\underline{$\PHfrappe{a_0}{c_0}{c_1}$}}\only<-4,6->{$\PHfrappe{a_0}{c_0}{c_1}$}~\PHconcat~
%\only<6>{\underline{$\PHfrappe{b_0}{d_0}{d_1}$}}\only<-5,7->{$\PHfrappe{b_0}{d_0}{d_1}$}~\PHconcat~
%\only<7>{\underline{$\PHfrappe{c_1}{b_0}{b_1}$}}\only<-6,8->{$\PHfrappe{c_1}{b_0}{b_1}$}~\PHconcat~
%\only<8>{\underline{$\PHfrappe{b_1}{d_1}{d_2}$}}\only<-7,9->{$\PHfrappe{b_1}{d_1}{d_2}$}}
%\end{center}
%
%\end{frame}
%
%
%
%\begin{frame}
%  \frametitle{Over- and Under-approximations}
%  \framesubtitle{\cite{PMR12-MSCS}}
%
%Static analysis by abstractions:
%\begin{fleches}
%  \item Directly checking an objective sequence $R$ is hard
%  \item Rather check the approximations $P$ and $Q$, where \tval{$P \Rightarrow R \Rightarrow Q$}:
%\end{fleches}
%
%\begin{center}
%\scalebox{0.5}{
%\figsa
%}
%\end{center}
%
%\only<-7>{~}
%\only<8->{
%Linear w.r.t.~the number of sorts and \\exponential w.r.t.~the number of processes in each sort
%\begin{fleches}
%  \item Efficient for big models with few levels of expression
%\end{fleches}
%}
%\end{frame}
%
%\begin{frame}
%  \frametitle{Under-approximation}
%
%\def \tu {3}
%\def \tub {4}
%\def \tuf {5}
%
%\begin{columns}
%\begin{column}{0.48\textwidth}
%\begin{center}
%\scalebox{0.55}{
%\begin{tikzpicture}
%\exatt
%\TState{-\tu}{a_1,b_1,c_1,d_0}
%\TState{\tub-}{a_0,b_1,c_0,d_0}
%\node[process,very thick] (d_2) at (d_2.center) {?};
%\end{tikzpicture}
%}
%\end{center}
%
%\end{column}
%\begin{column}{0.52\textwidth}
%
%\f New abstract structure
%
%\tval{Sufficient condition}:
%\smallskip
%
%\only<2->{
%\smallskip
%\begin{itemize}
%  \item \only<-\tu>{no cycle} \only<\tub->{\sout{no cycle}}
%  \item \only<-\tu>{each objective has a solution} \only<\tub->{\sout{each objective has a solution}}
%\end{itemize}
%\begin{center}
%  \only<\tu>{\Large\textcolor{darkgreen}{$R$ is \textbf{true}}} \only<\tuf>{\Large\textcolor{darkyellow}{\textbf{Inconclusive}}}
%\end{center}
%}
%
%\end{column}
%\end{columns}
%
%\only<-\tu>{
%\sauyes
%}
%
%\only<\tub->{
%\sauinconc
%}
%
%\end{frame}
%
%
%
%\begin{frame}
%  \frametitle{Over-approximation}
%
%\def \to {4}
%\def \tob {5}
%\def \tof {6}
%\def \tokp {7}
%
%\begin{columns}
%\begin{column}{0.48\textwidth}
%\begin{center}
%\scalebox{0.55}{
%\begin{tikzpicture}
%\exatt
%\TState{-\to}{a_1,b_0,c_0,d_1}
%\TState{\tob-}{a_1,b_1,c_1,d_0}
%\node[process,very thick] (d_2) at (d_2.center) {?};
%\end{tikzpicture}
%}
%\end{center}
%\bigskip
%
%\end{column}
%\begin{column}{0.52\textwidth}
%
%\tval{Necessary condition}:
%
%\medskip
%\only<2->{
%\only<3-\to>{\sout{There exists a traversal}}\only<2,\tob->{There exists a traversal}
%with no cycle
%
%\smallskip
%\begin{itemize}
%  \item \only<3-\to>{\sout{objective $\rightarrow$ follow one solution}}\only<1-2,\tob->{objective $\rightarrow$ follow one solution}
%  \item solution $\rightarrow$ follow all processes
%  \item process $\rightarrow$ follow all objectives
%\end{itemize}
%\begin{center}
%  \only<\to>{\Large\textcolor{red}{$R$ is \textbf{false}}}\only<\tof->{\Large\textcolor{darkyellow}{\textbf{Inconclusive}}}
%\end{center}
%}
%
%\end{column}
%\end{columns}
%
%%\bigskip
%
%\only<1-\to>{
%\saono
%}
%
%\only<\tob->{
%\saoinconc
%}
%
%\end{frame}
%
%\begin{frame}
%  \frametitle{Stochastic Features}
%  \framesubtitle{\cite{PMR10-TSE}}
%
%\begin{itemize}
%  \item Introduces time features
%  \item Parameters: either $(r,sa)$, or the \tval{firing interval} $[d;D]$.
%\end{itemize}
%
%\begin{columns}
%\begin{column}{0.5\textwidth}
%
%\only<2->{
%\scalebox{0.9}{
%\begin{tikzpicture}[plot,xscale=0.35,yscale=1]
%\draw[axe] (0,0) -- (15.5,0) node[right] {$t$};
%\draw[axe] (0,0) -- (0,1);
%\draw[ticks] (0,0) node[below] {$0$};
%\draw[mean] (4,0) -- (4,0.9) node[right]{$\frac{1}{r}$};
%\draw[dashed] (0.10127,0) -- (0.10127,0.9) node[right] {$d$} (14.75552,0) --
%(14.75552,0.9) node[left] {$D$};
%\pgfplothandlerlineto
%\pgfplotxyfile{plots/BioAtlanSTIC-0409/erlang-0.25-1.table}
%\pgfusepath{stroke}
%\draw[interval] (0.10127,0) -- (14.75552,0);
%\end{tikzpicture}
%}
%
%\scalebox{0.9}{
%\begin{tikzpicture}[plot,xscale=0.35,yscale=1]
%\draw[axe] (0,0) -- (15.5,0) node[right] {$t$};
%\draw[axe] (0,0) -- (0,1);
%\draw[ticks] (0,0) node[below] {$0$};
%\draw[mean] (4,0) -- (4,0.9) node[right]{$\frac{1}{r}$};
%\draw[dashed] (1.29879,0) -- (1.29879,0.9) node[right] {$d$} (8.19327,0) --
%(8.19327,0.9) node[left] {$D$};
%\pgfplothandlerlineto
%\pgfplotxyfile{plots/BioAtlanSTIC-0409/erlang-0.25-5.table}
%\pgfusepath{stroke}
%\draw[interval] (1.29879,0) -- (8.19327,0);
%\end{tikzpicture}
%}
%
%\scalebox{0.9}{
%\begin{tikzpicture}[plot,xscale=0.35,yscale=1]
%\draw[axe] (0,0) -- (15.5,0) node[right] {$t$};
%\draw[axe] (0,0) -- (0,1);
%\draw[ticks] (0,0) node[below] {$0$};
%\draw[mean] (4,0) -- (4,0.9) node[right]{$\frac{1}{r}$};
%\draw[dashed] (2.96888,0) -- (2.96888,0.9) node[left] {$d$} (5.18245,0) --
%(5.18245,0.9) node[right] {$D$};
%\pgfplothandlerlineto
%\pgfplotxyfile{plots/BioAtlanSTIC-0409/erlang-0.25-50.table}
%\pgfusepath{stroke}
%\draw[interval] (2.96888,0) -- (5.18245,0);
%\end{tikzpicture}
%}
%
%~~\textcolor{lightred}{\rule{17mm}{4.5pt}} ~ action duration
%}
%
%\end{column}
%\begin{column}{0.5\textwidth}
%\begin{center}
%
%\only<3->{
%\scalebox{0.9}{
%\begin{tikzpicture}
%\path[use as bounding box] (-1,-1) rectangle (2,1);
%\TSort{(0,0)}{a}{2}{l}
%\TSort{(2,0)}{b}{2}{r}
%\THit{a_0}{}{b_0}{.west}{b_1}
%\THit{a_0}{out=-120,in=180,selfhit}{a_0}{.west}{a_1}
%\path[bounce]
%\TBounce{a_0}{bend left}{a_1}{.south}
%\TBounce{b_0}{bend left}{b_1}{.south}
%;
%\TState{3-}{a_0,b_0}
%\end{tikzpicture}
%}
%
%\scalebox{0.9}{
%\begin{tikzpicture}[plot]
%\node[anchor=west] at (0,0.4) {$\PHfrappe{a_0}{b_0}{b_1}$};
%\draw[axe] (0,0) -- (4,0) node[right] {$t$};
%\draw (0,0.1) -- (0,-0.1);
%\draw[interval] (1.7,0) -- (3,0);
%
%\node[anchor=west] at (0,-0.6) {$\PHfrappe{a_0}{a_0}{a_1}$};
%\draw[axe] (0,-1) -- (4,-1) node[right] {$t$};
%\draw (0,-0.9) -- (0,-1.1);
%\draw[interval,orange] (0.2,-1) -- (1.5,-1);
%\end{tikzpicture}
%}
%
%\noindent
%\f $b_1$ reached with a \tval{very low probability}.
%}
%
%\end{center}
%\end{column}
%\end{columns}
%
%\pause[4]
%\bigskip
%\begin{fleches}
%  \item Tests by \textbf{simulation} and \textbf{model-checking}
%\end{fleches}
%
%\end{frame}
%
%
%

