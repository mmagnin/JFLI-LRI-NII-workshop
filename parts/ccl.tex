% Performances et conclusion


\begin{frame}[c]
  \frametitle{Implementation}

\small
\tval{Workflow}:
\begin{itemize}
  \item Read and translate the models with \tval{OCaml}\\
        \quad\f Uses the existing free library \tval{Pint}\\
        \quad\f Documentation + examples: \footnotesize\url{http://loicpauleve.name/pint}
  \item Express the problem in \tval{ASP} (logic programming)\\
        \quad\f Solve with \tval{Clingo} (\tval{Gringo} + \tval{Clasp})
\end{itemize}

\pause
\bigskip
\footnotesize
\begin{tabular}{c||r@{+}l|c|c||c|c||c|c|}
\multicolumn{5}{c||}{Model specifications} & \multicolumn{2}{c||}{IG inference} & \multicolumn{2}{c|}{Parameters inference}\\
\hline
\tval{Name} & S & CS & P & A & $\Delta t$ & Edges & $\Delta t$ & Parameters\\
\hline
  \tval{\ex{[EGFR20]}} & \tval{20} & 22 & 152 & 399 & \tval{1s} & 50 & \tval{1s} & 191\\
\hline
  \tval{\ex{[TCRSIG40]}} & \tval{40} & 14 & 156 & 301 & \tval{1s} & 54 & \tval{1s} & 143\\
\hline
  \tval{\ex{[TCRSIG94]}} & \tval{94} & 39 & 448 & 1124 & \tval{13s} & 169 & $\infty$ & $2.10^9$\\
\hline
  \tval{\ex{[EGFR104]}} & \tval{104} & 89~ & 748 & 2356 & \tval{4min} & 241 & \tval{1min 30s} & $1.10^6 / 2.10^6$\\
\hline
\end{tabular}

S = Sorts \quad CS = Cooperative sorts \quad P = Processes \quad A = Actions

\bigskip
\quad\tval{\ex{[EGFR20]}}: Epidermal Growth Factor Receptor, by \"Ozg\"ur Sahin et al.\\
\quad\tval{\ex{[EGFR104]}}: Epidermal Growth Factor Receptor, by Regina Samaga et al.\\
\quad\tval{\ex{[TCRSIG40]}}: T-Cell Receptor Signaling, by Steffen Klamt et al.\\
\quad\tval{\ex{[TCRSIG94]}}: T-Cell Receptor Signaling, by Julio Saez-Rodriguez et al.

\end{frame}



\begin{frame}[c]
  \frametitle{Summary}

\only<1>{
\begin{block}{Process Hitting and ASP}
\begin{itemize}
  \item Inference of the \tval{complete Interaction Graph}
  \item Inference of the \tval{possibly partial Parametrization}
  \item Enumerate all full \& \tval{admissible Parametrizations}
\end{itemize}
\quad\quad\f Exhaustive approaches
\end{block}

\bigskip
\tval{Complexity}: linear in the number of genes, exponential in the number of regulators of one gene

}

\only<2>{
\begin{block}{Contribution: new translation Process Hitting $\rightsquigarrow$ Ren\'e Thomas}
\begin{fleches}
  \item New \tval{formal link} between the two models
  \item More \tval{visibility} to the Process Hitting
  \item Inference approach that takes benefit from both the Process Hitting compact structure and the power of ASP 
\end{fleches}
\end{block}
}

\end{frame}

\frame{\frametitle{Further work}

\begin{alertblock}{Models and algorithms}
\begin{itemize}
\item Add \textbf{priorities} in the Process Hitting framework and adapt the static analyses approaches for this enriched model \cite{FolschetteCS2Bio13} 
\item From priorities to \textbf{quantitative timing information}
\item Connect Process Hitting compact structure with \textbf{decomposition techniques} in continuous approaches \cite{ammarreduction2012, ChancellorACMR13}
\end{itemize}
\end{alertblock}

\begin{block}{Application}
\begin{itemize}
\item Use the approach for the analysis of \textbf{larger} biological networks 
\item Contribute to the \textbf{discovery} of biological regulatory networks based on biological data 
\item Study key properties (e.g. concept of \textbf{resilience})
\end{itemize}
\end{block}

}
