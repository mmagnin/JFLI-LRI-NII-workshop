%% Application aux réseaux de régulation biologique
%
%\subsection{Presentation of Interaction Graphs}
%
%\begin{frame}
%  \frametitle{Application to Biological Systems}
%
%\bigskip
%\tval{The Process Hitting framework}:
%\begin{itemize}
%  \item new formalism with simple elements
%  \item dynamic modeling
%  \item efficient reachability checking
%  \item general framework that could be applied to...
%\end{itemize}
%
%\bigskip
%\tval{Biological systems}:
%\begin{itemize}
%  \item gene/protein couples
%  \item Thomas' Modeling: Interaction Graphs
%  \item hard to study \f use Process Hitting
%\end{itemize}
%
%\end{frame}
%
%
%
%\begin{frame}
%  \frametitle{Interaction Graphs}
%
%\begin{columns}
%\begin{column}{0.5\textwidth}
%
%\medskip
%\begin{itemize}
%  \item Nodes = Genes
%  \item Directed edges = Interactions
%\end{itemize}
%
%\end{column}
%\begin{column}{0.5\textwidth}
%
%\begin{flushright}
%\scalebox{1.2}{
%\begin{tikzpicture}[grn,node distance=2cm]
%\path[use as bounding box] (-0.6,-1) rectangle (2.3,0.5);
%\node(a) {a};
%\node(b) [right of=a] {b};
%\path
%  (a) edge[inh, bend left] node[elabel, above] {$-2$} (b)
%  (b) edge[inh, bend left] node[elabel, below=-2pt] {$-1$} (a)
%  (a) edge[act,loop left=10] node[elabel, left] {$+1$} (a);
%\path
%  node[elabel, below=-1em of a] {$0..2$}
%  node[elabel, below=-10pt of b] {$0..1$};
%%\node (ba) [elabel,below of=a,below=-2pt] {$0..2$} (a);
%%\node (bb) [elabel,below of=b] {$0..1$};
%\end{tikzpicture}
%}
%
%+ Parametrization \quad~
%\end{flushright}
%
%\end{column}
%\end{columns}
%
%\bigskip
%
%\only<2>{
%\Large
%\begin{flushright}
%\textcolor{couleurtheme}{State Graphs}\hspace{2em}~\!\!
%\end{flushright}
%\normalsize
%
%\begin{columns}
%\begin{column}{0.4\textwidth}
%
%\begin{itemize}
%  \item One level at a time
%  \item Asynchronous
%\end{itemize}
%
%\end{column}
%\begin{column}{0.6\textwidth}
%
%\begin{flushright}
%\begin{tikzpicture}
%\path[use as bounding box] (0,-2) rectangle (6,-0.5);
%\matrix at (0,0) [anchor=north west,column sep=.8cm, row sep=.8cm]
%{
%\node (s01) {$\PHetat{a_0,b_1}$}; \&
%\node (s11) {$\PHetat{a_1,b_1}$}; \&
%\node (s21) {$\PHetat{a_2,b_1}$}; \\
%\node (s00) {$\PHetat{a_0,b_0}$}; \&
%\node (s10) {$\PHetat{a_1,b_0}$}; \&
%\node (s20) {$\PHetat{a_2,b_0}$}; \\
%};
%\path[->]
%    (s00) edge (s10) edge (s01)
%    (s10) edge (s11) edge (s20)
%    (s21) edge (s11) edge (s20)
%;
%\end{tikzpicture}
%\end{flushright}
%
%\end{column}
%\end{columns}
%
%\bigskip
%Exponential number of states w.r.t.~the number of genes
%}
%\end{frame}
%

%\subsection{Translation from Thomas' framework to Process Hitting \& Refining}

\begin{frame}
  \frametitle{Translation of the Generalized Dynamics}

\begin{center}
Positive interaction:
\end{center}
\begin{columns}
\begin{column}{0.45\textwidth}

\begin{center}
\begin{tikzpicture}[grn,node distance=2cm]
\path[use as bounding box] (-0.8,-.5) rectangle (2.5,0.6);
\node (a) {a};
\node (b) [right of=a] {b};
\path (a) edge[act, bend left] node[elabel, above] {$+1$} (b);
\end{tikzpicture}
\end{center}

\end{column}
\begin{column}{0.05\textwidth}

\begin{flushright}
\medskip
\Large
\f
\normalsize
\end{flushright}

\end{column}
\begin{column}{0.5\textwidth}

\begin{center}
\scalebox{0.8}{
\begin{tikzpicture}
\path[use as bounding box] (-1,0) rectangle (5,1.5);
\TSort{(0,0.5)}{a}{2}{l}
\TSort{(4,0)}{b}{3}{r}

\only<2>{
\THit{a_1}{}{b_0}{.west}{b_1}
\THit{a_1}{}{b_1}{.west}{b_2}
\path[bounce,bend left]
\TBounce{b_0}{}{b_1}{.south west}
\TBounce{b_1}{}{b_2}{.south west};
}

\only<3->{
\THit{a_1}{ulhit}{b_0}{.west}{b_1}
\THit{a_1}{ulhit}{b_1}{.west}{b_2}
\path[bounce,bend left,pulhit]
\TBounce{b_0}{bulhit}{b_1}{.south west}
\TBounce{b_1}{bulhit}{b_2}{.south west};
\THit{a_0}{}{b_2}{.west}{b_1}
\THit{a_0}{}{b_1}{.west}{b_0}
\path[bounce,bend right]
\TBounce{b_2}{}{b_1}{.north west}
\TBounce{b_1}{}{b_0}{.north west};
}

\end{tikzpicture}
}
\end{center}

\end{column}
\end{columns}

\bigskip
\only<4->{
\begin{center}
Negative interaction:
\end{center}
\begin{columns}
\begin{column}{0.45\textwidth}

\begin{center}
\begin{tikzpicture}[grn,node distance=2cm]
\path[use as bounding box] (-0.8,-.5) rectangle (2.5,0.6);
\node (a) {a};
\node (b) [right of=a] {b};
\path (a) edge[inh, bend left] node[elabel, above] {$-1$} (b);
\end{tikzpicture}
\end{center}

\end{column}
\begin{column}{0.05\textwidth}

\begin{flushright}
\medskip
\Large
\f
\normalsize
\end{flushright}

\end{column}
\begin{column}{0.5\textwidth}

\begin{center}
\scalebox{0.8}{
\begin{tikzpicture}
\path[use as bounding box] (-1,0) rectangle (5,1.5);
\TSort{(0,0.5)}{a}{2}{l}
\TSort{(4,0)}{b}{3}{r}

\only<5>{
\THit{a_1}{}{b_2}{.west}{b_1}
\THit{a_1}{}{b_1}{.west}{b_0}
\path[bounce,bend right]
\TBounce{b_2}{}{b_1}{.north west}
\TBounce{b_1}{}{b_0}{.north west};
}

\only<6>{
\THit{a_1}{ulhit}{b_2}{.west}{b_1}
\THit{a_1}{ulhit}{b_1}{.west}{b_0}
\path[bounce,bend right,pulhit]
\TBounce{b_2}{bulhit}{b_1}{.north west}
\TBounce{b_1}{bulhit}{b_0}{.north west};
\THit{a_0}{}{b_0}{.west}{b_1}
\THit{a_0}{}{b_1}{.west}{b_2}
\path[bounce,bend left]
\TBounce{b_0}{}{b_1}{.south west}
\TBounce{b_1}{}{b_2}{.south west};
}

\end{tikzpicture}
}
\end{center}

\end{column}
\end{columns}
}

\end{frame}



%\begin{frame}
%  \frametitle{Refining with Actions Removal}
%
%\bigskip
%Prevent behaviors by deleting \tval{unrealistic actions}
%
%\bigskip
%\begin{center}
%\begin{tikzpicture}
%%\path[use as bounding box] (-1,0) rectangle (5,3);
%\TSort{(0,0.5)}{a}{2}{l}
%\TSort{(4,0)}{b}{3}{r}
%
%\THit{a_0}{ulhit}{b_2}{.west}{b_1}
%\THit{a_0}{ulhit}{b_1}{.west}{b_0}
%\path[bounce,bend right,pulhit]
%\TBounce{b_2}{bulhit}{b_1}{.north west};
%\path[bounce,bend right,pulhit]
%\TBounce{b_1}{bulhit}{b_0}{.north west};
%\THit{a_1}{}{b_0}{.west}{b_1}
%\path[bounce,bend left]
%\TBounce{b_0}{}{b_1}{.south west};
%
%\only<1>{
%\THit{a_1}{}{b_1}{.west}{b_2}
%\path[bounce,bend left]
%\TBounce{b_1}{}{b_2}{.south west};
%}
%
%\only<2>{
%\THit{a_1}{draw=red,fill=red}{b_1}{.west}{b_2}
%\path[bend left,bounce,fill=red]
%\TBounce{b_1}{draw=red}{b_2}{.south west};
%}
%
%\only<3>{}
%
%\end{tikzpicture}
%\end{center}
%
%\end{frame}



\begin{frame}
  \frametitle{Refining with Cooperation}

Allow \tval{cooperation} between two genes
\begin{itemize}
  \item How to express \ex{$\PHfrappe{(a_1 \wedge b_1)}{z_0}{z_1}$}?
%  \item TTT
\end{itemize}
\begin{fleches}
  \item Add a \textbf{cooperative sort} reflecting the state of $a$ and $b$
\end{fleches}

\begin{center}
\scalebox{0.8}{
\begin{tikzpicture}
%\path[use as bounding box] (-1,0) rectangle (5,3);
\TSort{(0,3.5)}{a}{2}{l}
\TSort{(0,0.5)}{b}{2}{l}
\TSort{(8,2)}{z}{2}{r}
\only<2->{
\TSetTick{ab}{0}{00}
\TSetTick{ab}{1}{01}
\TSetTick{ab}{2}{10}
\TSetTick{ab}{3}{11}
\TSort{(4,1)}{ab}{4}{r}
}

\only<1>{
\THit{a_1}{}{z_0}{.west}{z_1}
\THit{b_1}{}{z_0}{.west}{z_1}
\path[bounce,bend left]
\TBounce{z_0}{}{z_1}{.south west};
}

\only<3>{
\THit{a_1}{}{ab_0}{.west}{ab_2}
\THit{a_1}{}{ab_1}{.west}{ab_3}
\path[bounce,bend left]
\TBounce{ab_0}{}{ab_2}{.south west}
\TBounce{ab_1}{}{ab_3}{.south west};
}

\only<4->{
\THit{a_1}{ulhit}{ab_0}{.west}{ab_2}
\THit{a_1}{ulhit}{ab_1}{.west}{ab_3}
\path[bounce,bend left,pulhit]
\TBounce{ab_0}{bulhit}{ab_2}{.south west}
\TBounce{ab_1}{bulhit}{ab_3}{.south west};
}
\only<4>{
\THit{a_0}{}{ab_2}{.west}{ab_0}
\THit{a_0}{}{ab_3}{.west}{ab_1}
\path[bounce,bend right]
\TBounce{ab_2}{}{ab_0}{.north west}
\TBounce{ab_3}{}{ab_1}{.north west};
}

\only<5->{
\THit{a_0}{ulhit}{ab_2}{.west}{ab_0}
\THit{a_0}{ulhit}{ab_3}{.west}{ab_1}
\path[bounce,bend right,pulhit]
\TBounce{ab_2}{bulhit}{ab_0}{.north west}
\TBounce{ab_3}{bulhit}{ab_1}{.north west};
}
\only<5>{
\THit{b_1}{}{ab_0}{.west}{ab_1}
\THit{b_1}{}{ab_2}{.west}{ab_3}
\path[bounce,bend left]
\TBounce{ab_0}{}{ab_1}{.south west}
\TBounce{ab_2}{}{ab_3}{.south west};
}

\only<6->{
\THit{b_1}{ulhit}{ab_0}{.west}{ab_1}
\THit{b_1}{ulhit}{ab_2}{.west}{ab_3}
\path[bounce,bend left,pulhit]
\TBounce{ab_0}{bulhit}{ab_1}{.south west}
\TBounce{ab_2}{bulhit}{ab_3}{.south west};
}
\only<6>{
\THit{b_0}{}{ab_1}{.west}{ab_0}
\THit{b_0}{}{ab_3}{.west}{ab_2}
\path[bounce,bend right]
\TBounce{ab_1}{}{ab_0}{.north west}
\TBounce{ab_3}{}{ab_2}{.north west};
}

\only<7->{
\THit{b_0}{ulhit}{ab_1}{.west}{ab_0}
\THit{b_0}{ulhit}{ab_3}{.west}{ab_2}
\path[bounce,bend right,pulhit]
\TBounce{ab_1}{bulhit}{ab_0}{.north west}
\TBounce{ab_3}{bulhit}{ab_2}{.north west};
\THit{ab_3}{}{z_0}{.west}{z_1}
\path[bounce,bend left]
\TBounce{z_0}{}{z_1}{.south west};
}

\end{tikzpicture}
}
\end{center}
\only<8>{
\begin{fleches}
\item Introduces a temporal shift (over-approximation)
\end{fleches}
}
\end{frame}


\begin{frame}
  \frametitle{Using Process Hitting for\\Interaction Graphs Study}

\begin{block}{Motivation}
\begin{itemize}
  \item Interaction Graph is the \textbf{historical discrete model} (suitable and widespread in biological research) 
  \item Several tools exist of the analysis of interaction graphs, but the \textbf{state graph} is needed for some results $\Rightarrow$ \textbf{combinatorial explosion}
\end{itemize}  
\end{block}

\begin{block}{Contribution: Process Hitting to study large Biological Regulatory Networks}
\begin{itemize}
  \item Translation from Interaction Graphs + Refining
  \item \tval{Efficient static analysis}
\end{itemize}
\end{block}

\end{frame}

\begin{frame}[c]
  \frametitle{The Process Hitting modeling}

\begin{block}{Key features} 
\begin{itemize}
  \item \tval{Dynamic} modeling with an \tval{atomistic} point of view
  \begin{fleches}
    \item Independent actions
    \item Cooperation modeled with cooperative sorts
  \end{fleches}

  \item Efficient \tval{static analysis}
  \begin{fleches}
    \item Reachability of a process can be computed in \tval{linear time}\\
          \quad in the number of sorts
  \end{fleches}

  \item Useful for the study of \tval{large biological models}
  \begin{fleches}
    \item Up to hundreds of sorts
  \end{fleches}

  \end{itemize}
\end{block}

\begin{alertblock}{(Future) extensions} 
  \begin{itemize}
    \item Actions with stochasticity
    \item Actions with priorities
    \item Continuous time with clocks?
  \end{itemize}
\end{alertblock}

\end{frame}
