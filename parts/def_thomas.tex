% Définition du modèle de Thomas

\frame{\frametitle{Short introduction to Biological Regulatory Networks}

\begin{block}{Principle of R. Thomas' discrete modeling \cite{thomas1976ac}}
\begin{itemize}
\item \textbf{Activations} and \textbf{inhibitions} between genes
\item Gene/protein couples
\item Genes expression is associated to a set of \texthl{discrete logic} levels
\item \textbf{Effective control} beyond a given \texthl{threshold}; opposite effect below.
\end{itemize}
\end{block}

\begin{block}{Interaction graph} 
\vspace{-1.8em}

\begin{columns}
\begin{column}{0.5\textwidth}

\medskip
\begin{itemize}
  \item Nodes = \textbf{Genes}
  \item Directed edges = \textbf{Interactions}
  \item But what is the \textbf{evolutionary tendency} of $a$ when $a$ is at level $1$ and $b$ at level $1$? $\Rightarrow$ Need for \alert{\textbf{parametrization}}  %discrete parameters defining the strength of the interactions, based on the set of resources $\Rightarrow$ \textbf{Parametrization}  
\end{itemize}

\end{column}
\begin{column}{0.5\textwidth}

\begin{flushright}
\scalebox{1.2}{
\begin{tikzpicture}[grn,node distance=2cm]
\path[use as bounding box] (-0.6,-1) rectangle (2.3,0.5);
\node(a) {a};
\node(b) [right of=a] {b};
\path
  (a) edge[inh, bend left] node[elabel, above] {$2-$} (b)
  (b) edge[inh, bend left] node[elabel, below=-2pt] {$1-$} (a)
  (a) edge[act,loop left=10] node[elabel, left] {$1+$} (a);
\path
  node[elabel, below=-1em of a] {$0..2$}
  node[elabel, below=-10pt of b] {$0..1$};
%\node (ba) [elabel,below of=a,below=-2pt] {$0..2$} (a);
%\node (bb) [elabel,below of=b] {$0..1$};
\end{tikzpicture}
}
%
%+ Parametrization \quad~
\end{flushright}

\end{column}
\end{columns}

\end{block}
}

\begin{frame}
  \frametitle{Biological Regulatory Network (Thomas' modeling)}
%  \framesubtitle{\tcite{RCB08}}

\begin{tabular}{cccc}
%
\begin{tikzpicture}[grn]
\path[use as bounding box] (-0.7,-0.3) rectangle (2.5,2);
% Nœuds noirs
\only<1,3->{
  \node[inner sep=0] (z) at (2,0.75) {z};
  \node[inner sep=0] (a) at (0,1.5) {a};
  \node[inner sep=0] (b) at (0,0) {b};
  \path
    node[elabel, below=-1em of a] {$0..1$}
    node[elabel, below=-1em of b] {$0..1$}
    node[elabel, below=-1em of z] {$0..2$};
}
% Nœuds grisés
\only<2>{
  \node[inner sep=0,light] (z) at (2,0.75) {z};
  \node[inner sep=0,light] (a) at (0,1.5) {a};
  \node[inner sep=0,light] (b) at (0,0) {b};
  \path
    node[elabel, below=-1em of a,light] {$0..1$}
    node[elabel, below=-1em of b,light] {$0..1$}
    node[elabel, below=-1em of z,light] {$0..2$};}

%% Arcs colorés
\only<1,4->{\path
  (a) edge[inh,loop left=10] node[elabel, left] {$1-$} (a)
  (a) edge[act] node[elabel, above=-2pt] {$1+$} (z)
  (b) edge[inh] node[elabel, below=-2pt] {$1-$} (z);}
% Arcs grisés
\only<2-3>{\path
  (a) edge[inhgray,loop left=10] node[elabel, left,light] {$1-$} (a)
  (a) edge[actgray] node[elabel, above=-2pt,light] {$1+$} (z)
  (b) edge[inhgray] node[elabel, below=-2pt,light] {$1-$} (z);
}
\end{tikzpicture}
&%
\only<2-4>{\color{light}}
\begin{tabular}[b]{c|c|c}
  \multicolumn{2}{c|}{$\omega$} & \multirow{2}{*}{$k_{z, \omega}$} \\
\cline{1-2}
  $a$ & $b$ & \\
\hline
  $-$ & $+$ & $1$ \\
  $-$ & $-$ & $0$ \\
  \only<6>{\color{red}}$+$ & \only<6>{\color{red}}$+$ & \only<6>{\color{red}}$2$ \\
  $+$ & $-$ & $1$
\end{tabular}
%%\begin{tabular}[b]{c|c}
%%  $\omega$ & $k_{z, \omega}$ \\
%%\hline
%%  $\emptyset$ & $1$ \\
%%  $\{b\}$ & $0$ \\
%%  \only<7-8>{\color{red}}$\{a\}$ & \only<7-8>{\color{red}}$2$ \\
%%  $\{a;b\}$ & $1$
%%\end{tabular}
~~~~&
\only<2-4>{\color{light}}%
\begin{tabular}[b]{c|c}
  $\omega$ & \multirow{2}{*}{$k_{a, \omega}$} \\
\cline{1-1}
  $a$ & \\
\hline
  $+$ & $1$ \\
  $-$ & $0$
\end{tabular}
%%\begin{tabular}[b]{c|c}
%%  $\omega$ & $k_{a, \omega}$ \\
%%\hline
%%  $\emptyset$ & $1$ \\
%%  $\{a\}$ & $0$
%%\end{tabular}
&
\only<2-4>{\color{light}}%
\begin{tabular}[b]{cc}
  & $k_{b, \omega}$ \\
\cline{2-2}
  & $1$
\end{tabular}
%%\begin{tabular}[b]{c|c}
%%  $\omega$ & $k_{b, \omega}$ \\
%%\hline
%%  $\emptyset$ & $1$
%%\end{tabular}
\\
\only<2-4>{$\underbrace{\text{\hspace{3cm}}}_{\text{Interaction Graph}}$}%
&
\multicolumn{3}{r}{\only<5->{$\underbrace{\text{\hspace{6.5cm}}}_\text{Parametrization}$}}
\end{tabular}

\bigskip
% Historical model...
\only<1>{

\bigskip
Proposed by Ren\'e Thomas in 1973, several extensions since then
\medskip
\begin{liste}
  \item \tval{Historical bio-informatics model} for studying genes interactions
  \item Widely used and well-adapted to represent dynamic gene systems
\end{liste}
}

% Interaction Graph
\only<2-4>{
\tval{Interaction Graph}: structure of the system (genes \& interactions)

\pause[3]
\medskip
\begin{liste}
  \item \tval{Nodes}: genes
  \begin{itemize}
    \item Name \qex{$a$, $b$, $z$}
    \item Possible values (levels of expression) \qex{$0..1$, $0..2$}
  \end{itemize}
\pause[4]
  \item \tval{Edges}: interactions
  \begin{itemize}
    \item Threshold \qex{$1$}
    \item Type (activation or inhibition) \qex{$+$ / $-$}
  \end{itemize}
\end{liste}
}

% Parametrization
\only<5->{
\medskip
\tval{Parametrization}: strength of the influences (cooperations)

\medskip
\begin{liste}
  \item Maps of tendencies for each gene
  \begin{fleches}
    \item To any \tval{influences of predecessors} \qex{$\omega$}
    \item Corresponds a \tval{parameter} \qex{$k_{x,\omega}$}
  \end{fleches}
\pause[6]
\medskip
  \item \ex{$k_{z, \{a^+, b^+\}} = 2$} \quad means: \qex{$z$ tends to $2$ when $a \geq 1$ and $b < 1$}
\end{liste}
%\pause[10]
}
\end{frame}



\begin{frame}
  \frametitle{Biological Regulatory Network (Thomas' modeling)}
%  \framesubtitle{\tcite{RCB08}}

\begin{tabular}{cccc}

\begin{tikzpicture}[grn]
\path[use as bounding box] (-0.7,-0.3) rectangle (2.5,2);
\node[inner sep=0] (z) at (2,0.75) {z};
\node[inner sep=0] (a) at (0,1.5) {a};
\node[inner sep=0] (b) at (0,0) {b};
\path
  node[elabel, below=-1em of a] {$0..1$}
  node[elabel, below=-1em of b] {$0..1$}
  node[elabel, below=-1em of z] {$0..2$};
\path
  (a) edge[inh,loop left=10] node[elabel, left] {$1-$} (a)
  (a) edge[act] node[elabel, above=-2pt] {$1+$} (z)
  (b) edge[inh] node[elabel, below=-2pt] {$1-$} (z);
\end{tikzpicture}
&
\begin{tabular}[b]{c|c|c}
  \multicolumn{2}{c|}{$\omega$} & \multirow{2}{*}{$k_{z, \omega}$} \\
\cline{1-2}
  $a$ & $b$ & \\
\hline
  $-$ & $+$ & $1$ \\
  $-$ & $-$ & $0$ \\
  $+$ & $+$ & $2$ \\
  $+$ & $-$ & $1$
\end{tabular}
~~~~&
\begin{tabular}[b]{c|c}
  $\omega$ & \multirow{2}{*}{$k_{a, \omega}$} \\
\cline{1-1}
  $a$ & \\
\hline
  $+$ & $1$ \\
  $-$ & $0$
\end{tabular}
&
\begin{tabular}[b]{cc}
   & $k_{b, \omega}$ \\
\cline{2-2}
   & $1$
\end{tabular}
\\
\multicolumn{4}{r}{$\underbrace{\text{\hspace{10.1cm}}}_{\text{Biological Regulatory Network}}$}
\end{tabular}

\bigskip

\begin{itemize}
  \item[\f] All needed information to run the model or study its dynamics:
  \begin{itemize}\normalsize
    \item Build the State Graph
    \item Find reachability properties, fixed points, attractors
    \item Other properties...
  \end{itemize}
\end{itemize}

\begin{itemize}
  \item[\f] \tval{Strengths}: well adapted for the study of biological systems
  \item[\f] \tval{Drawbacks}: inherent complexity; needs the full\\
    \quad \quad specification of cooperations
\end{itemize}
\end{frame}
